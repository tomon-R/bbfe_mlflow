\newpage
\section{3次元スロッシング}

数値解析例として矩形緒層内スロッシング解析を実施し、参考文献の結果と比較した。

\subsection{解析条件}

物性値パラメータを表\ref{table:3d-sloshing-material-property}に示す。流体は水と空気を仮定したパラメータである。
\renewcommand{\arraystretch}{1}
\begin{table}[H]
	\centering
	\caption{物性値}
	\begin{tabular}{ccccccc}
		\hline
		Test case & $\rho_1$ & $\rho_2$ & $\mu_1$ & $\mu_2$ & $\mathrm{g}$ \\
		\hline 
		Case$1$ & $1000$ & $1.2$ & $1.0e-3$ & $1.8e-5$ & $9.8$ \\
		\hline         
	\end{tabular}
	\label{table:3d-sloshing-material-property}
\end{table}
\renewcommand{\arraystretch}{1.0}

解析パラメータを表\ref{table:3d-sloshing-parameter}に示す。メッシュの細かさの違う条件で解析し、精度検証した。
\renewcommand{\arraystretch}{1}
\begin{table}[H]
	\centering
	\caption{解析パラメータ}
	\begin{tabular}{cccccc}
		\hline
		Test case & $\Delta t$ & メッシュ幅$dx$ & 界面幅$D$ & 再初期化回数 & 再初期化$\Delta \tau$\\
		\hline 
		Case$1$ & $0.0025$ & $0.025$ & $0.075$ & $5$ & $0.0001$\\
		Case$2$ & $0.0010$ & $0.0125$ & $0.030$ & $5$ & $0.0001$\\
		\hline         
	\end{tabular}
	\label{table:3d-sloshing-parameter}
\end{table}
\renewcommand{\arraystretch}{1.0}

解析領域は幅$1.0\mathrm{m}$、奥行き$0.2\mathrm{m}$、高さ$1.2\mathrm{m}$であり、水が高さ$0.5\mathrm{m}$まで入っている状態である。
境界条件は全壁面でスリップ条件を与えた。
メッシュは六面体1次要素で、節点数18,081、要素数15,360である。

\begin{figure}[H]
	\centering
	\begin{minipage}[b]{0.49\columnwidth}
	    \centering
	    \includegraphics[width=6truecm]{pics/3d-sloshing/mesh.jpeg}
		\caption{3次元スロッシングの計算メッシュ}
		\label{fig:3d-sloshing-mesh}
	\end{minipage}
	\begin{minipage}[b]{0.49\columnwidth}
	    \centering
	    \includegraphics[width=6truecm]{pics/3d-sloshing/levelset_init.jpeg}
		\caption{3次元スロッシングのレベルセット関数($T=0$)}
		\label{fig:3d-sloshing-levelset_t0_3d}
	\end{minipage}
\end{figure}

スロッシングを起こすための水平加速度$f_{x}$は以下の式で与える。
\begin{equation}
	f_{x} = A \omega^2 \sin{\omega t}
\end{equation}
ここで、振幅$A=0.0093 \mathrm{m}$、角速度$\omega = 5.311 \mathrm{rad/s}$と設定した。

\newpage
\subsection{解析結果}
\subsubsection{メッシュ固定の場合の解析結果}

\begin{figure}[H]
	\centering
	\begin{minipage}[b]{0.19\columnwidth}
	    \centering
	    \includegraphics[width=3.5truecm]{pics/3d-sloshing/CN-dx0025/result_0010.jpeg}
	\end{minipage}
	\begin{minipage}[b]{0.19\columnwidth}
	    \centering
	    \includegraphics[width=3.5truecm]{pics/3d-sloshing/CN-dx0025/result_0012.jpeg}
	\end{minipage}
	\begin{minipage}[b]{0.19\columnwidth}
	    \centering
	    \includegraphics[width=3.5truecm]{pics/3d-sloshing/CN-dx0025/result_0014.jpeg}
	\end{minipage}
	\begin{minipage}[b]{0.19\columnwidth}
	    \centering
	    \includegraphics[width=3.5truecm]{pics/3d-sloshing/CN-dx0025/result_0016.jpeg}
	\end{minipage}
	\begin{minipage}[b]{0.19\columnwidth}
	    \centering
	    \includegraphics[width=3.5truecm]{pics/3d-sloshing/CN-dx0025/result_0018.jpeg}
	\end{minipage}
	\caption{スロッシングの結果($1.0\mathrm{s}$~$1.8\mathrm{s}$)}
	\label{fig:sloshing-result}
\end{figure}
\begin{figure}[H]
	\centering
	\begin{minipage}[b]{0.19\columnwidth}
	    \centering
	    \includegraphics[width=3.5truecm]{pics/3d-sloshing/CN-dx0025/result_0090.jpeg}
	\end{minipage}
	\begin{minipage}[b]{0.19\columnwidth}
	    \centering
	    \includegraphics[width=3.5truecm]{pics/3d-sloshing/CN-dx0025/result_0092.jpeg}
	\end{minipage}
	\begin{minipage}[b]{0.19\columnwidth}
	    \centering
	    \includegraphics[width=3.5truecm]{pics/3d-sloshing/CN-dx0025/result_0094.jpeg}
	\end{minipage}
	\begin{minipage}[b]{0.19\columnwidth}
	    \centering
	    \includegraphics[width=3.5truecm]{pics/3d-sloshing/CN-dx0025/result_0096.jpeg}
	\end{minipage}
	\begin{minipage}[b]{0.19\columnwidth}
	    \centering
	    \includegraphics[width=3.5truecm]{pics/3d-sloshing/CN-dx0025/result_0098.jpeg}
	\end{minipage}
	\caption{スロッシングの結果($9.0\mathrm{s}$~$9.8\mathrm{s}$)}
	\label{fig:sloshing-result}
\end{figure}

解析結果の左端の水位の時間変化のグラフを図~\ref{fig:3d-sloshing-result}に示す。
流体メッシュは固定して慣性加速度を与えた場合と、流体メッシュに加速度を与えて移動させた場合(ALE)の結果を比較し、どちらも近い結果が得られた。
さらにメッシュを粗いケースと細かいケースで計算したところ、細かくすることによって文献の値により近い結果が得られた。
\begin{figure}[H]
    \centering
	\includegraphics[width=15truecm]{pics/3d-sloshing/height_time.pdf}
	\caption{スロッシング解析結果の左端の水位と参考文献\cite{Okamoto1992}, \cite{Sakuraba2001}との比較}
	\label{fig:3d-sloshing-result}
\end{figure}

参考までに、スロッシング解析では衝撃捕捉項(\ref{sec:fluid}章の式(\ref{fluid-GLS}))を入れない場合、6.0sあたりから計算が不安定になり発散しやすい傾向が見られた。
実際にALE法ではあるが、shock capturing項(衝撃捕捉項)を入れない場合と入れる場合の計算を比較し、入れない場合は6.0s過ぎた付近から計算が不安定となり、その後発散するが、入れた場合は安定的に計算ができた結果が報告されている\cite{Sakuraba1999}。

\newpage
\subsubsection{メッシュを移動させた場合の解析結果(ALE記述の実装の確認)}
ここでは、メッシュを移動させた場合の結果を示す。

\begin{figure}[H]
	\centering
	\begin{minipage}[b]{0.19\columnwidth}
	    \centering
	    \includegraphics[width=3.5truecm]{pics/3d-sloshing/ALE-dx0025/result_0010.jpeg}
	\end{minipage}
	\begin{minipage}[b]{0.19\columnwidth}
	    \centering
	    \includegraphics[width=3.5truecm]{pics/3d-sloshing/ALE-dx0025/result_0012.jpeg}
	\end{minipage}
	\begin{minipage}[b]{0.19\columnwidth}
	    \centering
	    \includegraphics[width=3.5truecm]{pics/3d-sloshing/ALE-dx0025/result_0014.jpeg}
	\end{minipage}
	\begin{minipage}[b]{0.19\columnwidth}
	    \centering
	    \includegraphics[width=3.5truecm]{pics/3d-sloshing/ALE-dx0025/result_0016.jpeg}
	\end{minipage}
	\begin{minipage}[b]{0.19\columnwidth}
	    \centering
	    \includegraphics[width=3.5truecm]{pics/3d-sloshing/ALE-dx0025/result_0018.jpeg}
	\end{minipage}
	\caption{スロッシングの結果($1.0\mathrm{s}$~$1.8\mathrm{s}$)}
	\label{fig:sloshing-result}
\end{figure}
\begin{figure}[H]
	\centering
	\begin{minipage}[b]{0.19\columnwidth}
	    \centering
	    \includegraphics[width=3.5truecm]{pics/3d-sloshing/ALE-dx0025/result_0090.jpeg}
	\end{minipage}
	\begin{minipage}[b]{0.19\columnwidth}
	    \centering
	    \includegraphics[width=3.5truecm]{pics/3d-sloshing/ALE-dx0025/result_0092.jpeg}
	\end{minipage}
	\begin{minipage}[b]{0.19\columnwidth}
	    \centering
	    \includegraphics[width=3.5truecm]{pics/3d-sloshing/ALE-dx0025/result_0094.jpeg}
	\end{minipage}
	\begin{minipage}[b]{0.19\columnwidth}
	    \centering
	    \includegraphics[width=3.5truecm]{pics/3d-sloshing/ALE-dx0025/result_0096.jpeg}
	\end{minipage}
	\begin{minipage}[b]{0.19\columnwidth}
	    \centering
	    \includegraphics[width=3.5truecm]{pics/3d-sloshing/ALE-dx0025/result_0098.jpeg}
	\end{minipage}
	\caption{スロッシングの結果($9.0\mathrm{s}$~$9.8\mathrm{s}$)}
	\label{fig:sloshing-result}
\end{figure}

\newpage
\subsection{解析結果(並列化計算)}

領域分割法による並列計算結果についても検証を行った。
図~\ref{fig:3d-sloshing-mesh-parallel16}に領域分割数16の場合の計算メッシュの分割結果を示す。
節点数は23571, 要素数は15360である。

\begin{figure}[H]
    \centering
    \includegraphics[width=6truecm]{pics/3d-sloshing-parallel/mesh_parallel16.jpeg}
	\caption{スロッシングの計算メッシュ(領域分割数16の場合)}
	\label{fig:3d-sloshing-mesh-parallel16}
\end{figure}

図~\ref{fig:3d-sloshing-result-parallel}に並列数2,4, 8, 16の場合の計算結果を示す。
並列数を変えた場合でも逐次計算の結果と一致した結果が得られた。

\begin{figure}[H]
    \centering
	\includegraphics[width=15truecm]{pics/3d-sloshing-parallel/sloshing_result_comparison_parallel.pdf}
	\caption{スロッシング解析結果の左端の水位と参考文献\cite{Okamoto1992}, \cite{Sakuraba2001}との比較}
	\label{fig:3d-sloshing-result-parallel}
\end{figure}

図~\ref{fig:3d-sloshing-parallel-speedup}に並列化による加速率効率のグラフを示す。
\begin{figure}[H]
    \centering
	\includegraphics[width=15truecm]{pics/3d-sloshing-parallel/sloshing_parallel_speedup.jpg}
	\caption{スロッシング解析の並列化効率}
	\label{fig:3d-sloshing-parallel-speedup}
\end{figure}